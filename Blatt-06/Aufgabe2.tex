\documentclass[mathserif, xcolor=divipsnames]{beamer}
\usepackage{ngerman,amsmath,amssymb}
\usepackage[utf8]{inputenc}
\usepackage{xcolor}
\usepackage[ngerman]{babel}
\usepackage{tikz,rotating}
\usetikzlibrary{shapes,calc}
\tikzset{
  treenode/.style = {align=center, inner sep=0pt, text centered,
    font=\sffamily}
}
\clubpenalty=10000
\widowpenalty=10000
\displaywidowpenalty=10000
\usetheme{Ilmenau}
\usefonttheme{structuresmallcapsserif}
\setbeamertemplate{navigation symbols}{}
\title{Übungsblatt 06\\
Aufgabe 2 (Extensive Spiele)}
\author{Tim Blome, Michael Koller, Ayleen Schinko}
\date{02. Dezember 2016}
\begin{document}
\begin{frame}
  \maketitle
\end{frame}

\section{Aufgabe 2 a}
\begin{frame}
  \footnotesize a) Zeigen Sie, dass es kein Normalformspiel gegen kann, in dem es kein
  pareto-optimalen Ausgang gibt.\\
  \vspace{0.2cm}
  \normalsize
  \only<1>{\begin{Definition}[Strategisches Spiel (Normalform)]
    Ein strategisches Spiel $G = \left< N, (A_i)_{i\in N}, (u_i)_{i\in N}
    \right>$ besteht aus
    \begin{itemize}
      \item einer endlichen Menge $N = 1,\dots , n$ an Spielern,
      \item einer Menge an Aktion $A_i$ für jeden Spieler,
      \item einer Nutzenfunktion (utility) $u_i : A_1 \times \dots \times A_n
        \rightarrow \mathbb{R}$ für jeden Spieler.
    \end{itemize}
  \end{Definition}}
  \only<2>{\begin{Definition}[pareto-optimal]
    Ein Ausgang $o^*$ ist pareto-optimal, wenn er von keinem anderen Ausgang
    pareto-dominiert wird.
  \end{Definition}
  Wenn $o$ und $o'$ Ausgänge eines Spieles eines, sodass $o$ für alle Spieler
  mindestens so gut ist wie $o'$ und $o$ von einem Spieler echt bevorzugt wird,
  dann pareto-dominiert $o$ $o'$.}
  \only<3>{\small Wir gehen davon aus, dass für jeden Spieler die Aktionsmenge endlich
  und nicht leer ist.\\
  Da für jeden Spieler die Nutzenfunktion einen Wert aus $\mathbb{R}$ liefert
  und die $\leq$-Ordnung eine totale Ordnung auf $\mathbb{R}$ bildet, kann für
  jede Komponente eines Ausgangs ein eindeutiger Vergleich in dieser Komponente
  zu jedem anderen Ausgang hergestellt werden.\\
  Widerspruchsbeweis: Angenommen: Es gibt ein Normalformspiel, in dem es
  keinen pareto-optimalen Ausgang gibt.\\
  So müssen alle Ausgänge einen Zyklus bilden oder eine unendliche besser
  werdend Kette.\\
  Da auf den Ausgängen eine partielle Ordnung existiert, kann kein Zyklus
  in den Ausgangstupeln möglich sein. Die unendliche besser werdende Kette ist
  nicht möglich, da das Spiel endlich ist, weil alle Aktionsmenge der Spieler
  als endlich angenommen wurden.\\
  Es muss also in der Tupelmenge der Ausgänge mindestens ein lokales Maximum
  existieren, dass nach Definition dann pareto-optimal ist.
  }
\end{frame}

\section{Aufgabe 2 b}
\begin{frame}[fragile]
  \footnotesize b) Angenommen, zwei Spieler teilen 50 (gleichwertige) Münzen auf. Der Nutzen
  von Spielern ist die Menge von erhaltenen Münzen. Zuerst kann Spieler 1 die
  Münzen aufteilen, danach Spieler 2 den Anteil wählen. Bestimmen Sie ein
  Nash-Gleichgewicht mittels Rückwärtsinduktion.\\
  \vspace{0.5cm}
  \normalsize
  \begin{tikzpicture}[level/.style = {sibling distance = 35mm/#1, level
    distance = 2cm}]
    \node [treenode,ellipse,draw] (A1) {Agent 1\\$(25,25)$}
    child{ node [treenode,ellipse,draw] (A21) {Agent 2\\$(50-x,x)$}
      child{ node (E1) {$(50-x,x)$}}
      child{ node (E2) {$(x,50-x)$}}}
    child{ node [treenode,ellipse,draw] (A22) {Agent 2\\$(25,25)$}
      child{ node (E3) {$(25,25)$}}
      child{ node (E4) {$(25,25)$}}}
    child{ node [treenode,ellipse,draw] (A23) {Agent 2\\$(x,50-x)$}
      child{ node (E5) {$(50-x,x)$}}
      child{ node (E6) {$(x,50-x)$}}};

      \path (A1) -- (A21) node[near start, above left] {$x > 50-x$};
      \path [draw=orange] (A1) -- (A22) node[near start] {$x = 50-x$};
      \path (A1) -- (A23) node[near start, above right] {$x < 50-x$};
      \path [draw=orange] (A21) -- (E1) node[near start, left] {$x$};
      \path (A21) -- (E2) node[near start, right] {$50-x$};
      \path [draw=orange] (A22) -- (E3) node[near start, left] {$x$};
      \path [draw=orange] (A22) -- (E4) node[near start, right] {$50-x$};
      \path (A23) -- (E5) node[near start, left] {$x$};
      \path [draw=orange] (A23) -- (E6) node[near start, right] {$50-x$};
\end{tikzpicture}
\end{frame}

\section{Aufgabe 2 c}
\begin{frame}[fragile]
  \footnotesize c) Nennen Sie ein teilspielperfektes Equilibrium dieses
  Spieles.\\
  \vspace{0.5cm}
  \normalsize
  \begin{tikzpicture}[level/.style = {sibling distance = 180mm - 680mm/3 *
    #1 + 100mm * #1 * #1 - 40mm/3 * #1 * #1 * #1, level distance = 1.5cm}]
    \node [treenode,circle,draw] (A11) {1}
    child{ node (A21) {}
      child{ node (E1) {$(6,6)$}}
      child{ node (E2) {$(10,2)$}}}
    child{ node (A22) {}
      child{ node (E3) {$(2,10)$}}
      child{ node [treenode,circle,draw] (A12) {1}
        child{ node (A23) {}
          child{ node (E4) {$(15,15)$}}
          child{ node (E5) {$(20,10)$}}}
        child{ node (A24) {}
          child{ node (E6) {$(10,20)$}}
          child{ node (E7) {$(15,15)$}}}}};

      \node [treenode,ellipse,draw,text width=3.5cm,fill=white] (B1) at ($(A21)!0.5!(A22)$) {2};
      \node [treenode,ellipse,draw,text width=3.5cm,fill=white] (B2) at ($(A23)!0.5!(A24)$) {2};

      \path (A11) -- (A21) node[near start, above left] {W};
      \path [draw=orange] (A11) -- (A22) node[near start, above right] {N};
      \path [draw=orange] (A21) -- (E1) node[near start, left] {W};
      \path (A21) -- (E2) node[near start, right] {N};
      \path (A22) -- (E3) node[near start, left] {W};
      \path [draw=orange] (A22) -- (A12) node[near start, right] {N};
      \path [draw=orange] (A12) -- (A23) node[near start, left] {W};
      \path (A12) -- (A24) node[near start, right] {N};
      \path [draw=orange] (A23) -- (E4) node[near start, left] {W};
      \path (A23) -- (E5) node[near start, right] {N};
      \path [draw=orange] (A24) -- (E6) node[near start, left] {W};
      \path (A24) -- (E7) node[near start, right] {N};
\end{tikzpicture}
\end{frame}

\section{Aufgabe 2 d}
\begin{frame}
  \footnotesize d) Begründen Sie, warum in einem extensiven Spiel mit perfekter
  Information ein durch Rückwärtsinduktion gefundener Ausgang ein
  Nash-Gleichgewicht sein muss.\\
  \vspace{0.2cm}
  \normalsize
  In einem extensiven Spiel mit perfekter Information wird durch
  Rückwärtsinduktion in jedem Teilspiel ein teilspielperfektes Equilibrium
  gefunden. Keiner der Akteure könnte sich in einer Situation anders
  entscheiden um seine Situation zu verbessern.\\
  Das so enstehende teilspielperfekte Equilibrium für das gesamte Spiel ist
  also gleichzeitig auch ein Nash-Gleichgewicht.
\end{frame}

\section{Aufgabe 2 e}
\begin{frame}[fragile]
  \footnotesize e) 1) Stellen Sie Fußball/Komödie als UVEF-Spiel dar.\\
  \vspace{0.5cm}
  \normalsize
  \begin{tikzpicture}[level/.style = {sibling distance = 180mm - 680mm/3 *
    #1 + 100mm * #1 * #1 - 40mm/3 * #1 * #1 * #1, level distance = 1.5cm}]
    \node [treenode,ellipse,draw] (Ada) {Ada}
    child{ node (B1) {}
      child{ node (E1) {$(2,1)$}}
      child{ node (E2) {$(0,0)$}}}
    child{ node (B2) {}
      child{ node (E3) {$(0,0)$}}
      child{ node (E4) {$(1,2)$}}};

      \node [treenode,ellipse,draw,text width=3.5cm,fill=white] (Bert) at
      ($(B1)!0.5!(B2)$) {Bert};

      \path (Ada) -- (B1) node[near start, above left] {f};
      \path (Ada) -- (B2) node[near start, above right] {k};
      \path (B1) -- (E1) node[near start, left] {f};
      \path (B1) -- (E2) node[near start, right] {k};
      \path (B2) -- (E3) node[near start, left] {f};
      \path (B2) -- (E4) node[near start, right] {k};
\end{tikzpicture}
\end{frame}

\begin{frame}[fragile]
  \footnotesize e) 2) Stellen Sie zweifaches Gefangendilemma als UVEF-Spiel dar.\\
  \vspace{0.5cm}
  \normalsize
  \begin{tikzpicture}[level/.style = {sibling distance = - 1mm/6 * #1 * #1 *#1
    + 6mm * #1 * #1 - 251mm/6 * #1 + 86mm , level
    distance = 1.4cm}]
    \node [treenode,circle,draw] (A11) {1}
    child{ node (A21) {}
      child{ node (A12) {}
        child{ node (A23) {}
          child{ node (E1) {\tiny\rotatebox{-90}{$(-2,-2)$}}}
          child{ node (E2) {\tiny\rotatebox{-90}{$(-5,-1)$}}}}
        child{ node (A24) {}
          child{ node (E3) {\tiny\rotatebox{-90}{$(-1,-5)$}}}
          child{ node (E4) {\tiny\rotatebox{-90}{$(-4,-4)$}}}}}
      child{ node (A13) {}
        child{ node (A25) {}
          child{ node (E5) {\tiny\rotatebox{-90}{$(-5,-1)$}}}
          child{ node (E6) {\tiny\rotatebox{-90}{$(-8,\phantom{-}0)$}}}}
        child{ node (A26) {}
          child{ node (E7) {\tiny\rotatebox{-90}{$(-4,-4)$}}}
          child{ node (E8) {\tiny\rotatebox{-90}{$(-7,-4)$}}}}}}
    child{ node (A22) {}
      child{ node (A14) {}
        child{ node (A27) {}
          child{ node (E9) {\tiny\rotatebox{-90}{$(-1,-5)$}}}
          child{ node (E10) {\tiny\rotatebox{-90}{$(-4,-7)$}}}}
        child{ node (A28) {}
          child{ node (E11) {\tiny\rotatebox{-90}{$(\phantom{-}0,-8)$}}}
          child{ node (E12) {\tiny\rotatebox{-90}{$(-3,-7)$}}}}}
      child{ node (A15) {}
        child{ node (A29) {}
          child{ node (E13) {\tiny\rotatebox{-90}{$(-4,-4)$}}}
          child{ node (E14) {\tiny\rotatebox{-90}{$(-7,-3)$}}}}
        child{ node (A30) {}
          child{ node (E15) {\tiny\rotatebox{-90}{$(-3,-7)$}}}
          child{ node (E16) {\tiny\rotatebox{-90}{$(-6,-6)$}}}}}};

      \node [treenode,ellipse,draw,text width=4cm,fill=white] (Azwei1) at
      ($(A21)!0.5!(A22)$) {2};
      \node [treenode,ellipse,draw,text width=6cm,fill=white] (Aeins) at
      ($(A13)!0.5!(A14)$) {1};
      \node [treenode,ellipse,draw,text width=7cm,fill=white] (Azwei2) at
      ($(A26)!0.5!(A27)$) {2};

      \path (A11) -- (A21) node[near start, above left] {c};
      \path (A11) -- (A22) node[near start, above right] {d};
      \path (A21) -- (A12) node[near start, left] {c};
      \path (A21) -- (A13) node[near start, right] {d};
      \path (A22) -- (A14) node[near start, left] {c};
      \path (A22) -- (A15) node[near start, right] {d};
      \path (A12) -- (A23) node[near start, left] {c};
      \path (A12) -- (A24) node[near start, right] {d};
      \path (A13) -- (A25) node[near start, left] {c};
      \path (A13) -- (A26) node[near start, right] {d};
      \path (A14) -- (A27) node[near start, left] {c};
      \path (A14) -- (A28) node[near start, right] {d};
      \path (A15) -- (A29) node[near start, left] {c};
      \path (A15) -- (A30) node[near start, right] {d};
      \path (A23) -- (E1) node[near start, left] {c};
      \path (A23) -- (E2) node[near start, right] {d};
      \path (A24) -- (E3) node[near start, left] {c};
      \path (A24) -- (E4) node[near start, right] {d};
      \path (A25) -- (E5) node[near start, left] {c};
      \path (A25) -- (E6) node[near start, right] {d};
      \path (A26) -- (E7) node[near start, left] {c};
      \path (A26) -- (E8) node[near start, right] {d};
      \path (A27) -- (E9) node[near start, left] {c};
      \path (A27) -- (E10) node[near start, right] {d};
      \path (A28) -- (E11) node[near start, left] {c};
      \path (A28) -- (E12) node[near start, right] {d};
      \path (A29) -- (E13) node[near start, left] {c};
      \path (A29) -- (E14) node[near start, right] {d};
      \path (A30) -- (E15) node[near start, left] {c};
      \path (A30) -- (E16) node[near start, right] {d};
\end{tikzpicture}
\end{frame}
\end{document}

\documentclass[mathserif, xcolor=divipsnames]{beamer}
\usepackage{ngerman,amsmath,amssymb}
\usepackage[utf8]{inputenc}
\usepackage{xcolor}
\usepackage[ngerman]{babel}
\usepackage{tikz,rotating}
\usetikzlibrary{shapes,calc}
\tikzset{
  treenode/.style = {align=center, inner sep=0pt, text centered,
    font=\sffamily}
}
\clubpenalty=10000
\widowpenalty=10000
\displaywidowpenalty=10000
\usetheme{Ilmenau}
\usefonttheme{structuresmallcapsserif}
\setbeamertemplate{navigation symbols}{}
\title{Übungsblatt 06\\
Aufgabe 2 (Extensive Spiele)}
\author{Tim Blome, Michael Koller, Ayleen Schinko}
\date{02. Dezember 2016}
\begin{document}
\begin{frame}
  \maketitle
\end{frame}

\section{Aufgabe 2 a}
\begin{frame}
  \footnotesize a) Zeigen Sie, dass es kein Normalformspiel gegen kann, in dem es kein
  pareto-optimalen Ausgang gibt.\\
  \vspace{0.2cm}
  \normalsize
  \only<1>{\begin{Definition}[Strategisches Spiel (Normalform)]
    Ein strategisches Spiel $G = \left< N, (A_i)_{i\in N}, (u_i)_{i\in N}
    \right>$ besteht aus
    \begin{itemize}
      \item einer endlichen Menge $N = 1,\dots , n$ an Spielern,
      \item einer Menge an Aktion $A_i$ für jeden Spieler,
      \item einer Nutzenfunktion (utility) $u_i : A_1 \times \dots \times A_n
        \rightarrow \mathbb{R}$ für jeden Spieler.
    \end{itemize}
  \end{Definition}}
  \only<2>{\begin{Definition}[pareto-optimal]
    Ein Ausgang $o^*$ ist pareto-optimal, wenn er von keinem anderen Ausgang
    pareto-dominiert wird.
  \end{Definition}
  Wenn $o$ und $o'$ Ausgänge eines Spieles eines, sodass $o$ für alle Spieler
  mindestens so gut ist wie $o'$ und $o$ von einem Spieler echt bevorzugt wird,
  dann pareto-dominiert $o$ $o'$.\\
  \vspace{0.5cm}
  nach Definition:\\
  Wenn für alle Spieler $o \geq o'$ (und für mindestens einen $\neq$), dann
  pareto-dominiert $o$ $o'$.
  }
  \only<3>{
  \begin{itemize}
    \item Jede Aktionsmenge eines Spielers endlich und nicht leer
    \item Durch Nutzenfunktion totale Ordnung auf             Aktionsmenge
  \end{itemize}

  Widerspruchsbeweis:\\
  Existiert Normalformspiel ohne Pareto optimalen Ausgang, so

  \begin{itemize}
    \item Bilden alle Ausgänge einen Zyklus
    \begin{itemize}
      \item[\boldmath$\rightarrow$] Zyklus nicht möglich, da partielle Ordnung
    \end{itemize}
    \item Bilden Ausgänge unendlich bessere werdende Kette
    \begin{itemize}
      \item[\boldmath$\rightarrow$] Kette nicht möglich, da Aktionsmenge endlich
    \end{itemize}
  \end{itemize}


  $\Rightarrow$ Es existiert lokales Maximum\\
  $\Rightarrow$ Pareto optimaler Ausgang
  }
\end{frame}

\section{Aufgabe 2 b}
\begin{frame}[fragile]
  \footnotesize b) Angenommen, zwei Spieler teilen 50 (gleichwertige) Münzen auf. Der Nutzen
  von Spielern ist die Menge von erhaltenen Münzen. Zuerst kann Spieler 1 die
  Münzen aufteilen, danach Spieler 2 den Anteil wählen. Bestimmen Sie ein
  Nash-Gleichgewicht mittels Rückwärtsinduktion.
  $$ y=50-x $$\\
  \normalsize
  \begin{center}
  \begin{tikzpicture}[level/.style = {sibling distance = 35mm/#1, level
    distance = 2cm}]
    \node [treenode,ellipse,draw] (A1) {Agent 1\\ \only<5->{$(25,25)$}}
    child{ node [treenode,ellipse,draw] (A21) {Agent 2\\ \only<2->{$(y,x)$}}
      child{ node (E1) {$(y,x)$}}
      child{ node (E2) {$(x,y)$}}}
    child{ node [treenode,ellipse,draw] (A22) {Agent 2\\ \only<3->{$(25,25)$}}
      child{ node (E3) {$(25,25)$}}
      child{ node (E4) {$(25,25)$}}}
    child{ node [treenode,ellipse,draw] (A23) {Agent 2\\ \only<4->{$(x,y)$}}
      child{ node (E5) {$(y,x)$}}
      child{ node (E6) {$(x,y)$}}};

      \path (A1) -- (A21) node[near start, above left] {$x > y$};
      \only<1-4>{\path (A1) -- (A22) node[near start] {$x = y$};}
      \only<5->{\path [draw=orange, line width=2pt] (A1) -- (A22) node[near start] {$x = y$};}
      \path (A1) -- (A23) node[near start, above right] {$x < y$};
      \only<1>{\path (A21) -- (E1) node[near start, left] {$x$};}
      \only<2->{\path [draw=orange, line width=2pt] (A21) -- (E1) node[near start, left] {$x$};}
      \path (A21) -- (E2) node[near start, right] {$y$};
      \only<1-2>{\path (A22) -- (E3) node[near start, left] {$x$};}
      \only<3->{\path [draw=orange, line width=2pt] (A22) -- (E3) node[near start, left] {$x$};}
      \only<1-2>{\path (A22) -- (E4) node[near start, right] {$y$};}
      \only<3->{\path [draw=orange, line width=2pt] (A22) -- (E4) node[near start, right] {$y$};}
      \path (A23) -- (E5) node[near start, left] {$x$};
      \only<1-3>{\path (A23) -- (E6) node[near start, right] {$y$};}
      \only<4->{\path [draw=orange, line width=2pt] (A23) -- (E6) node[near start, right] {$y$};}
  \end{tikzpicture}
  \end{center}
\end{frame}

\section{Aufgabe 2 c}
\begin{frame}[fragile]
  \footnotesize c) Nennen Sie ein teilspielperfektes Equilibrium dieses
  Spieles.\\
  \vspace{0.5cm}
  \normalsize
  \begin{center}
  \begin{tikzpicture}[level/.style = {sibling distance = 180mm - 680mm/3 *
    #1 + 100mm * #1 * #1 - 40mm/3 * #1 * #1 * #1, level distance = 1.5cm}]
    \node [treenode,circle,draw] (A11) {1}
    child{ node (A21) {}
      child{ node (E1) {$(6,6)$}}
      child{ node (E2) {$(10,2)$}}}
    child{ node (A22) {}
      child{ node (E3) {$(2,10)$}}
      child{ node [treenode,circle,draw] (A12) {1}
        child{ node (A23) {}
          child{ node (E4) {$(15,15)$}}
          child{ node (E5) {$(20,10)$}}}
        child{ node (A24) {}
          child{ node (E6) {$(10,20)$}}
          child{ node (E7) {$(15,15)$}}}}};

      \path (A11) -- (A21) node[near start, above left] {W};
      \only<1-6>{\path (A11) -- (A22) node[near start, above right] {N};}
      \only<7->{\path [draw=orange, line width=2pt] (A11) -- (A22) node[near start, above right] {N};}
      \only<1-5>{\path (A21) -- (E1) node[near start, left] {W};}
      \only<6->{\path [draw=orange, line width=2pt] (A21) -- (E1) node[near start, left] {W};}
      \path (A21) -- (E2) node[near start, right] {N};
      \path (A22) -- (E3) node[near start, left] {W};
      \only<1-4>{\path (A22) -- (A12) node[near start, right] {N};}
      \only<5->{\path [draw=orange, line width=2pt] (A22) -- (A12) node[near start, right] {N};}
      \only<1-3>{\path (A12) -- (A23) node[near start, left] {W};}
      \only<4->{\path [draw=orange, line width=2pt] (A12) -- (A23) node[near start, left] {W};}
      \path (A12) -- (A24) node[near start, right] {N};
      \only<1-2>{\path (A23) -- (E4) node[near start, left] {W};}
      \only<3->{\path [draw=orange, line width=2pt] (A23) -- (E4) node[near start, left] {W};}
      \path (A23) -- (E5) node[near start, right] {N};
      \only<1>{\path (A24) -- (E6) node[near start, left] {W};}
      \only<2->{\path [draw=orange, line width=2pt] (A24) -- (E6) node[near start, left] {W};}
      \path (A24) -- (E7) node[near start, right] {N};

      \node [treenode,ellipse,draw,text width=3.5cm,fill=white] (B1) at ($(A21)!0.5!(A22)$) {2};
      \node [treenode,ellipse,draw,text width=3.5cm,fill=white] (B2) at ($(A23)!0.5!(A24)$) {2};

      % \node [above of=A11] {$(15,15)$};
  \end{tikzpicture}
  \end{center}
\end{frame}

\section{Aufgabe 2 d}
\begin{frame}
  \footnotesize d) Begründen Sie, warum in einem extensiven Spiel mit perfekter
  Information ein durch Rückwärtsinduktion gefundener Ausgang ein
  Nash-Gleichgewicht sein muss.\\
  \vspace{0.2cm}
  \normalsize
  In solchem Spiel wird durch Rückwärtsinduktion in jedem Teilspiel ein
  teilspielperfektes Equilibrium gefunden.\\
  \vspace{0.3cm}
  $\Rightarrow$ Kein Akteur kann sich in Situation durch Umentscheiden
  verbessern\\
  \vspace{0.3cm}
  $\Rightarrow$ Teilspielperfektes Equilibrium für gesamtes Spiel ist
  Nash-Gleichgewicht
\end{frame}

\section{Aufgabe 2 e}
\begin{frame}[fragile]
  \footnotesize e) 1) Stellen Sie Fußball/Komödie als UVEF-Spiel dar.\\
  \vspace{0.5cm}
  \normalsize
  \begin{center}
  \begin{tikzpicture}[level/.style = {sibling distance = 180mm - 680mm/3 *
    #1 + 100mm * #1 * #1 - 40mm/3 * #1 * #1 * #1, level distance = 1.5cm}]
    \node [treenode,ellipse,draw] (Ada) {Ada}
    child{ node (B1) {}
      child{ node (E1) {$(2,1)$}}
      child{ node (E2) {$(0,0)$}}}
    child{ node (B2) {}
      child{ node (E3) {$(0,0)$}}
      child{ node (E4) {$(1,2)$}}};

      \node [treenode,ellipse,draw,text width=3.5cm,fill=white] (Bert) at
      ($(B1)!0.5!(B2)$) {Bert};

      \path (Ada) -- (B1) node[near start, above left] {f};
      \path (Ada) -- (B2) node[near start, above right] {k};
      \path (B1) -- (E1) node[near start, left] {f};
      \path (B1) -- (E2) node[near start, right] {k};
      \path (B2) -- (E3) node[near start, left] {f};
      \path (B2) -- (E4) node[near start, right] {k};
  \end{tikzpicture}
  \end{center}
\end{frame}

\begin{frame}[fragile]
  \footnotesize e) 2) Stellen Sie zweifaches Gefangendilemma als UVEF-Spiel dar.\\
  \vspace{0.5cm}
  \normalsize
  \begin{center}
  \begin{tikzpicture}[level/.style = {sibling distance = - 1mm/6 * #1 * #1 *#1
    + 6mm * #1 * #1 - 251mm/6 * #1 + 86mm , level
    distance = 1.4cm}]
    \node [treenode,circle,draw] (A11) {1}
    child{ node (A21) {}
      child{ node [treenode,circle,draw] (A12) {1}
        child{ node (A23) {}
          child{ node (E1) {\tiny\rotatebox{-90}{$(-2,-2)$}}}
          child{ node (E2) {\tiny\rotatebox{-90}{$(-5,-1)$}}}}
        child{ node (A24) {}
          child{ node (E3) {\tiny\rotatebox{-90}{$(-1,-5)$}}}
          child{ node (E4) {\tiny\rotatebox{-90}{$(-4,-4)$}}}}}
      child{ node [treenode,circle,draw] (A13) {1}
        child{ node (A25) {}
          child{ node (E5) {\tiny\rotatebox{-90}{$(-5,-1)$}}}
          child{ node (E6) {\tiny\rotatebox{-90}{$(-8,\phantom{-}0)$}}}}
        child{ node (A26) {}
          child{ node (E7) {\tiny\rotatebox{-90}{$(-4,-4)$}}}
          child{ node (E8) {\tiny\rotatebox{-90}{$(-7,-4)$}}}}}}
    child{ node (A22) {}
      child{ node [treenode,circle,draw] (A14) {1}
        child{ node (A27) {}
          child{ node (E9) {\tiny\rotatebox{-90}{$(-1,-5)$}}}
          child{ node (E10) {\tiny\rotatebox{-90}{$(-4,-7)$}}}}
        child{ node (A28) {}
          child{ node (E11) {\tiny\rotatebox{-90}{$(\phantom{-}0,-8)$}}}
          child{ node (E12) {\tiny\rotatebox{-90}{$(-3,-7)$}}}}}
      child{ node [treenode,circle,draw] (A15) {1}
        child{ node (A29) {}
          child{ node (E13) {\tiny\rotatebox{-90}{$(-4,-4)$}}}
          child{ node (E14) {\tiny\rotatebox{-90}{$(-7,-3)$}}}}
        child{ node (A30) {}
          child{ node (E15) {\tiny\rotatebox{-90}{$(-3,-7)$}}}
          child{ node (E16) {\tiny\rotatebox{-90}{$(-6,-6)$}}}}}};

      \path (A11) -- (A21) node[near start, above left] {c};
      \path (A11) -- (A22) node[near start, above right] {d};
      \path (A21) -- (A12) node[near start, left] {c};
      \path (A21) -- (A13) node[near start, right] {d};
      \path (A22) -- (A14) node[near start, left] {c};
      \path (A22) -- (A15) node[near start, right] {d};
      \path (A12) -- (A23) node[near start, left] {c};
      \path (A12) -- (A24) node[near start, right] {d};
      \path (A13) -- (A25) node[near start, left] {c};
      \path (A13) -- (A26) node[near start, right] {d};
      \path (A14) -- (A27) node[near start, left] {c};
      \path (A14) -- (A28) node[near start, right] {d};
      \path (A15) -- (A29) node[near start, left] {c};
      \path (A15) -- (A30) node[near start, right] {d};
      \path (A23) -- (E1) node[near start, left] {c};
      \path (A23) -- (E2) node[near start, right] {d};
      \path (A24) -- (E3) node[near start, left] {c};
      \path (A24) -- (E4) node[near start, right] {d};
      \path (A25) -- (E5) node[near start, left] {c};
      \path (A25) -- (E6) node[near start, right] {d};
      \path (A26) -- (E7) node[near start, left] {c};
      \path (A26) -- (E8) node[near start, right] {d};
      \path (A27) -- (E9) node[near start, left] {c};
      \path (A27) -- (E10) node[near start, right] {d};
      \path (A28) -- (E11) node[near start, left] {c};
      \path (A28) -- (E12) node[near start, right] {d};
      \path (A29) -- (E13) node[near start, left] {c};
      \path (A29) -- (E14) node[near start, right] {d};
      \path (A30) -- (E15) node[near start, left] {c};
      \path (A30) -- (E16) node[near start, right] {d};

      \node [treenode,ellipse,draw,text width=4cm,fill=white] (Azwei1) at
      ($(A21)!0.5!(A22)$) {2};
      \node [treenode,ellipse,draw,text width=1cm,fill=white] (Azwei21) at
      ($(A23)!0.5!(A24)$) {2};
      \node [treenode,ellipse,draw,text width=1cm,fill=white] (Azwei22) at
      ($(A25)!0.5!(A26)$) {2};
      \node [treenode,ellipse,draw,text width=1cm,fill=white] (Azwei23) at
      ($(A27)!0.5!(A28)$) {2};
      \node [treenode,ellipse,draw,text width=1cm,fill=white] (Azwei24) at
      ($(A29)!0.5!(A30)$) {2};
  \end{tikzpicture}
  \end{center}
\end{frame}
\end{document}
